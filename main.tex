\documentclass[journal, onecolumn]{IEEEtran}
\IEEEoverridecommandlockouts
% The preceding line is only needed to identify funding in the first footnote. If that is unneeded, please comment it out.
\usepackage{cite}
\usepackage{amsmath,amssymb,amsfonts}
\usepackage{algorithmic}
\usepackage{graphicx}
\usepackage{textcomp}
\usepackage{xcolor}
\usepackage{hyperref}
\usepackage{graphicx}
\graphicspath{ {./assets/} }

\newcommand{\tomegg}{
  \href{http://tome.gg}{Tome.gg}
}

\def\BibTeX{{\rm B\kern-.05em{\sc i\kern-.025em b}\kern-.08em
    T\kern-.1667em\lower.7ex\hbox{E}\kern-.125emX}}

    \begin{document}

\title{Tome.gg Whitepaper \\
{
  % \footnotesize \textsuperscript{*}Note: Sub-titles are not captured in Xplore and should not be used
}
% \thanks{Identify applicable funding agency here. If none, delete this.}
}

\author{\IEEEauthorblockN{1\textsuperscript{st} Darren Karl Sapalo}
\IEEEauthorblockA{\textit{Tome.gg (Founder)} 
Makati City, Philippines \\
darren.sapalo@gmail.com}
% \and
% \IEEEauthorblockN{2\textsuperscript{nd} Given Name Surname}
% \IEEEauthorblockA{\textit{dept. name of organization (of Aff.)} \\
% \textit{name of organization (of Aff.)}\\
% City, Country \\
% email address or ORCID}
% \and
% \IEEEauthorblockN{3\textsuperscript{rd} Given Name Surname}
% \IEEEauthorblockA{\textit{dept. name of organization (of Aff.)} \\
% \textit{name of organization (of Aff.)}\\
% City, Country \\
% email address or ORCID}
% \and
% \IEEEauthorblockN{4\textsuperscript{th} Given Name Surname}
% \IEEEauthorblockA{\textit{dept. name of organization (of Aff.)} \\
% \textit{name of organization (of Aff.)}\\
% City, Country \\
% email address or ORCID}
% \and
% \IEEEauthorblockN{5\textsuperscript{th} Given Name Surname}
% \IEEEauthorblockA{\textit{dept. name of organization (of Aff.)} \\
% \textit{name of organization (of Aff.)}\\
% City, Country \\
% email address or ORCID}
% \and
% \IEEEauthorblockN{6\textsuperscript{th} Given Name Surname}
% \IEEEauthorblockA{\textit{dept. name of organization (of Aff.)} \\
% \textit{name of organization (of Aff.)}\\
% City, Country \\
% email address or ORCID}
}

\maketitle
\begin{abstract}
  This whitepaper introduces the \tomegg platform, which aims to revolutionize software engineering education through hyper-personalized, inclusive, and accessible mentorships and apprenticeships. Emphasizing data ownership, the paper explores the use of personal growth repositories, enabling individuals to take control of their learning journey.
  
  The whitepaper also examines the potential of text-based Language Learning Model (LLM) AIs in enhancing mentorship experiences by processing and analyzing textual data, thereby providing tailored guidance. By addressing these innovations, the paper envisions a transformative educational landscape for software engineers, fostering an inclusive learning environment for all.
  
  In section \ref{sec:state_of_education}, the current state of education, software engineering challenges, and personalized education characteristics are discussed. Section \ref{sec:problems_challenges} defines the problems and challenges addressed by the platform. Section \ref{sec:stakeholders} describes the platform's stakeholders, customers, and beneficiaries. Section \ref{sec:tomegg} introduces Tome.gg, its vision, and the target community. Finally, section \ref{sec:contribution} presents the tools and services provided by the Tome.gg platform for users and contributors.
  \end{abstract}

\begin{IEEEkeywords}
learning, growth, personalized education, technology, software engineering, 
apprenticeship, mentorship, knowledge management
\end{IEEEkeywords}

\section{The current state of education}
\label{sec:state_of_education}
\subsection{The global scene}
With the COVID19 pandemic, rising inflation rates, and resource shortages due to 
threats of war, poverty and education are gravely affected.
The UN highlighted in their 2022 SDG Report \cite{b1} that the progress on global poverty rate has been reversed by
3 years, but could be as worse as poverty reduction efforts being reversed to 
as much as 9 years for low-income countries\cite{b2}. It is estimated that 
``147 million children missed more than half of their in-class instruction over 
the past two years [and this] generation of children could lose a combined total of 
\$17 trillion in lifetime earnings in present value'' \cite{b3}.

\subsection{Traditional education and academia}
With the lockdown caused by the pandemic, the modes of education changed rapidly to 
adapt to the needs of students and teachers. We saw technological advancements 
accelerate such as the necessity to evolve into using digital-first approaches, the
mass adoption of learning management systems, the mass production of digital content 
sold and made accessible online such as books, PDFs, online video recordings of 
software courses. 
Traditional education such as schools and universities eventually grew to adopt 
new communication channels such as online real-time classes, with audio and video data. 
With the movement to the digital space, both teachers and students learned the new 
rules of engagement and discovered challenges that did not exist pre-pandemic due to
the evolution of doing lectures, engaging students, managing students' attention, 
cheating, skipping classes, plagiarism and more.
With the recent development of AI LLM tools such as ChatGPT, Bing Chat, Bard, and 
Llama, university administrators and professors find themselves in a challenging
new educational landscape where the misuse of these widely accessible tools are 
threatening the growth, literacy, and numeracy of many young individuals.

\subsection{Software Engineering and Professional Development}
\label{sec:edu_materials}
Educational material for software engineers can be classified into two categories
 based on accessibility: online or offline. The enumerated list below is not exhaustive or complete, 
 but it provides a general idea of the current available educational materials for software engineers.

\textbf{Educational material accessible online:}

\begin{enumerate}
  \item Online-capable traditional educational institutions such as Higher Education Universities, or Research Institutes
  \item Video learning platforms (Pluralsight, Udemy)
  \item Certification platforms for proprietary tools (AWS, GCP, Azure)
  \item Official public documentation of a tool, service, or application
  \item Official discussion forum of a tool, service, or application (disqus)
  \item Question and answer portals (e.g. Quora) 
  \item Error repositories and indexes (e.g. Stack Overflow)
  \item Public and private communities (Substack, newsletters, Discord)
  \item Independent publishers (professional blogs)
  \item Topic- or concept-focused websites (e.g. microservices.io, regex101.com, roadmap.sh)
  \item Coaching and mentoring platforms (paid, free services)
  \item Public code repositories with discussions on issues and tickets
\end{enumerate}

\textbf{Educational material accessible offline:}

\begin{enumerate}
  \item Traditional universities, research institutes, or higher-education 
  universities that offer degree programs in computer science, engineering,
  or data science
  \item Events in varying formats - such as workshops, seminars, master classes, hackathons, casual meets etc.
  \item Events by organizer - such as corporate-backed organizations 
  (Microsoft Philippines, Google Philippines), University student-led organizations (Peer tutoring orgs), 
  Independent communities and organizations niches by tool, product, or practice 
  (e.g. Raspberry Pi enthusiasts, Golang Philippines, Node.js Philippines, Hasura community)
\end{enumerate}

The above section describes the various entrypoints of a student in software engineering
to enter into tech, whether through formal or informal sources. A person new to the industry
would have this broad variety of options which can be challenging and intimidating to 
navigate and discern which are relevant to a person's level of capabilities. 

\subsection{AI-assisted tooling for educational purposes}
The mass-market availability of AI Large Language Models (LLMs) such as ChatGPT, 
Bing Chat, Bard, and Llama enables people to access vast amounts of information 
and reasoning capabilities through natural languages (e.g., English) as an 
interface. These AI tools are trained on the majority of web data prior to 
2021\footnote{Need to cite paper on ChatGPT, other AI papers}, which include
publicly-available online educational material such as public documentation of
tools, services, APIs, and source code. 

Specifically for software engineers, these tools provide great answer 
approximations not just for generating code simply from a prompt, but also for 
explaining intimidating concepts. 

However, there are limitations to these AI models. For instance, AIs trained on 
older data may generate outdated or invalid responses due to changes in the field 
since the last training. Additionally, responses can sometimes produce AI 
hallucinations, which may be caused by statistical biases or inherent biases in 
the trained data, such as myths, gossip, or human misconceptions.

Working with LLMs has opened up a new domain of study called Prompt Engineering, 
which focuses on 'prompts,' the primary way of interacting with text-based AI 
tools. This groundbreaking innovation in technology enables software engineers 
and no-code or low-code developers to achieve greater capabilities that were 
previously inaccessible due to the required understanding of programming 
languages, tools, and services, which demanded more education, discipline, or 
broader context in software engineering.
\subsection{Adult development theories}

With the focus of education of software engineers, we explore some theories of 
adult development, education and learning.

% As a collective society, the human race faces the great challenges of poverty, 
% inequality, and inaccessible resources. \tomegg believes that education is the
% critical component towards empowerment, upliftment of life quality, and 
% equitable opportunities in a global scene. This sets the stage that necessitates
% global collaboration on this global crisis. 

% Who are affected? Who can contribute? What part should I play? How might we
% engage various stakeholders and contributors like inventors, educators, and 
% entrepreneurs? 
% The key to enabling this global collaboration would be effective communications
% and a reliable protocol to ensure clarity and understanding. 
% The research and design of this protocol is led by \tomegg. We introduce Tome.gg
% in the following section.


\begin{figure*}[t]
  \centering
  \includegraphics[width=\textwidth]{00-tomegg-actors.png}
  \caption{Example caption.}
  \label{fig:example}
\end{figure*}

\subsection{Personalized education}

In the Philippines, some universities have a ratio of one (1) teacher to sixty (60)
students. These ratios are not far from other developing countries.

\section{Problems and Challenges} 
\label{sec:problems_challenges}

\subsection{Barriers to education}

There are a variety of barriers to education:

\begin{enumerate}
  \item Financial costs - Quality education sometimes require a large sum of 
  money to access great teachers, educators, professors. In  2021 at one of the best universities to study 
  technology and software engineering, Massachusetts Institute of Technology (MIT) provided estimates of student 
  charges for an undergraduate course at around 77,020.00 USD: 72.5\% of that amount is intended for tuition 
  (estimated at 55,878.00 USD) and 27.5\% is for other costs like books, on-campus room \& board (estimated at 21,142.00 USD) \cite{b8}.
  From 2016 to 2021, MIT's tuition increased by 15.3\%, rising from 48,452.00 USD to 55,878.00 USD. \cite{b8}.
  
  A non-exhaustive list of the large expenses that universities spend on are: (a) salaries for administrators, professors, researchers, 
  and curriculum designers, (b) quality school equipment including maintenance and access to in-person libraries 
  and electronic libraries, (c) instructional costs to deliver the academic programs to the students, (d) 
  research and collaborations with industry partners to understand the growing demands of the job market. 
  A recent graduate of a software engineering degree commonly has the goal of acquiring a good, reliable job that 
  opens them to a broad selection of economic and growth opportunities.
  Having understood that, higher education institutions (HEIs) like MIT provide career advising services, host recruiter activities, and
  provide networking opportunities with established alumni from different companies.  
  For these reasons, many parents and students are willing to spend large amounts of money and 
  put their trust in these HEIs.

  to have a well-organized organizations or corporations operate and accrue the 
  reputation of effective evaluators in their fields (academe for university,
  proprietary software for large tech corporations like Amazon and Google).
  \item Signal to noise - With the broad selection of educational materials 
  accessible to software engineers enumerated in \ref{sec:edu_materials}, 
  the search for learning content that is appropriate for one's learning 
  can become overwhelming and intimidating. 
  
  Apprentices or junior software engineers may not be familiar yet with terminologies
  and concepts in software engineering to be able to quickly skim research abstracts or
  blog articles. In their inexperience, they might find themselves spending more time and
  energy in comprehending things with a 0 to 100\% expectation because they don't have a map to guide in 
  their navigation. This is reinforced by the proliferation of YouTube trend
  of videos with titles like 'Day in a life of a  software engineer' wherein
  software engineers simply talk about their daily life.
  \item Energy costs - 
  \item Opportunity costs - 
  \item Personalization costs - There is a wide variety of options and modes 
  of teaching/learning available online. Some categorize the modes of learning as:
  visual, auditory, kinesthetic \footnote{citation needed}. These learning 
  preferences are not mutually exclusive (i.e. you only learn effectively in one
  form). It is more likely that a person has a mixed preference of the different
  learning modalities. Aside from the percentages of preferences among the three,
  there is also the preference of frequency and transition speed, more specifically
  how smoothly the transitions
  between learning modalities are. When receiving 
  educational material that does not match your  is within the reach of 
  software engineers, which sometimes makes it overwhelming and intimidating. 
  \item Context switching costs for personalization - 
\end{enumerate}

\subsection{Property loss, theft, and attribution}

\section{Community stakeholders}
\label{sec:stakeholders}

The previous section described the apprenticeship model which uses the 
skill proficiency dimension to classify individuals.
The following section uses the behavior dimension to classify stakeholders
that we engage with in \tomegg whether individuals or organizations.

\subsection{Students, Learners, Apprentices}

These are people who have an area of interest that they wish to learn
about. This could be games, software engineering, medicine, law, or the like.
Often times these are younger people, or fresh graduates, or people who are
new to the workforce. In these cases, a young student would have the abundance of 
time, energy, interest, and curiosity.

\subsection{Teachers, Educators, Mentors}

These are people who have accumulated an abundance of knowledge in their field.
This could be people who have spent roughly 5-10 years in their domain or industry.
In these cases, they have an abundance of experience and context about their
field.

\subsection{Masters, Experts, PhDs}

These are people who have not only acquired knowledge in their field for decades,
but are actively involved in advancing the production and creation of new 
knowledge through research. This could be academic researchers, research scientists,
or even e-sports gaming professionals that are crafting or innovating the next meta of 
some video game. In these cases, they have an abundance of experience, context, and
data about their field that enables them to theory craft, to perform pattern analysis,
and to research new knowledge.

\subsection{Consumers, Afficionados, Connoisseurs, Savants}

These are people who may not be performers or creators in their field (e.g. cooking,
creating YouTube videos, or creating art) but are well-versed in the taste and 
quality of creations. They spend a lot of their time consuming different kinds of
things to ensure their evaluations and analysis remains sharp. In these cases, 
they have an abundance of observational or secondary experience in their field,
and focus on enabling the accessibility of learning about theories and patterns 
that either they or masters have developed.

\subsection{Creators, Curators, Performers}

These are individuals who are at the center of their field, actively participating,
and are directly involved in creating a new performance, developing a new creation, 
or remixing past things into something new. Some masters are also creators and 
performers, but not all creators are masters. In a similar regard, these people can
be considered as major manufacturers or creators, whether in the digital or physical 
sense. In these cases, these individuals have an abundance of primary experience in
their field and an immense abundance in their network, by having an incredible
connection with their audiences or markets, or the partnerships they make.

\subsection{Corporations, Organizations, Governments}

These are organizations which operate at greater velocity as compared to individuals.
Each organization has their own directions depending on what their objectives are.
They have the great opportunity of funneling their resources into the causes that
are important to them (e.g. profit, social impact, advocacies, inclusion) but with
this immense power and influence comes also greater opportunity cost and risk. As
such, they unfortunately cannot make as fast decisions as an individual might be 
able to, as they have people and systems that depend on them already that they risk
endangering or losing. In these cases, organizations are abundant in their resources,
financial credibility, brand identity (if any), networks, and assets.


\section{Tome.gg}
\label{sec:tomegg}
The paper is inclined to focus on software engineers as its primary beneficiaries because of 
their inclination to be learning-oriented and  first-adopters of new technologies.
However, the nature of technological evolution is that products and services are 
built rapidly to increase the wide adoption the new discoveries, which creates value
accessible to all. This is further discussed and generalized in section \ref{sec:stakeholders}.


Every single person in the world is playing a game. Some play their game casually,
while some play their game competitively. A person's game might be physical sports
where the objective is to win based on the rules of the game. A person's game
might be social interactions and their objective is to become popular with a huge 
following. A person's game might be business, and their objective is to make their 
numbers grow big. Everyone is playing a game.

  \textbf{Tome.gg is an educational and growth-oriented community that leverages
  lessons learned in games and applying them in one's personal and professional life.} 
  In machine learning under computer science, the concept of applying what was 
  learned from one domain or problem to another is called transfer learning\cite{b5}.
  At \tomegg, we use examples, theories, and concepts learned from games to explain
  and educate using analogies and comparisons. 
  For example, we can see a model of apprenticeship\cite{b4} applied in the gaming domain
  that represents the various stages of learning: people become apprentices, 
  journeymen, and masters.

\begin{figure}[t]
  \includegraphics[scale=0.7]{stakeholders-gaming}
  \centering
  \caption{In the gaming domain, the following terms are commonly used to refer 
  to people in relation to their skill level.}
  \label{fig:stakeholders-gaming} 
\end{figure}

\tomegg primarily focuses on mid- to senior-level software engineers. However, 
this focus on the software engineering domain and the focus on engineers at a
journeyman stage does not limit the opportunities offered to apprentices and 
masters. This focus does not restrict them from participating or contributing 
to this community. Everyone has a unique role to play in how to efficiently 
contribute in the global challenges in education, which we will discuss in 
section \ref{sec:contribution}.

The following section describes the stakeholders within and outside of the
\tomegg community.

\section{Contribution}
\label{sec:contribution}

\subsection{Abbreviations and Acronyms}\label{AA}
Define abbreviations and acronyms the first time they are used in the text, 
even after they have been defined in the abstract. Abbreviations such as 
IEEE, SI, MKS, CGS, ac, dc, and rms do not have to be defined. Do not use 
abbreviations in the title or heads unless they are unavoidable.

\section{Principles and Mindsets}
\label{sec:principles_mindsets}

\subsection{Inclusive education for all}

\subsection{Safety and freedom}

\subsection{Ownership and sovereignty}

\subsection{Openness and transparency}

\subsection{Lifelong learning}

\subsection{Sustainable growth}

Let tech do the heavy lifting.

\subsection{Pursuit of mastery and excellence}

\subsection{Actualizing one's dream}


\section{Community Directions}
\label{sec:community_directions}

\subsection{Building the community}

\subsection{Building the tome.gg Map}

\subsection{Collaboratively building an open pool of knowledge}

\subsection{Collecting an open pool of knowledge}


\subsection{Figures and Tables}
\paragraph{Positioning Figures and Tables} Place figures and tables at the top and 
bottom of columns. Avoid placing them in the middle of columns. Large 
figures and tables may span across both columns. Figure captions should be 
below the figures; table heads should appear above the tables. Insert 
figures and tables after they are cited in the text. Use the abbreviation 
``Fig.~\ref{fig}'', even at the beginning of a sentence.

\begin{table}[htbp]
\caption{Table Type Styles}
\begin{center}
\begin{tabular}{|c|c|c|c|}
\hline
\textbf{Table}&\multicolumn{3}{|c|}{\textbf{Table Column Head}} \\
\cline{2-4} 
\textbf{Head} & \textbf{\textit{Table column subhead}}& \textbf{\textit{Subhead}}& \textbf{\textit{Subhead}} \\
\hline
copy& More table copy$^{\mathrm{a}}$& &  \\
\hline
\multicolumn{4}{l}{$^{\mathrm{a}}$Sample of a Table footnote.}
\end{tabular}
\label{tab1}
\end{center}
\end{table}

\begin{figure}[htbp]
\centerline{\includegraphics{fig1.png}}
\caption{Example of a figure caption.}
\label{fig}
\end{figure}

Figure Labels: Use 8 point Times New Roman for Figure labels. Use words 
rather than symbols or abbreviations when writing Figure axis labels to 
avoid confusing the reader. As an example, write the quantity 
``Magnetization'', or ``Magnetization, M'', not just ``M''. If including 
units in the label, present them within parentheses. Do not label axes only 
with units. In the example, write ``Magnetization (A/m)'' or ``Magnetization 
\{A[m(1)]\}'', not just ``A/m''. Do not label axes with a ratio of 
quantities and units. For example, write ``Temperature (K)'', not 
``Temperature/K''.

\section*{Acknowledgment}

The preferred spelling of the word ``acknowledgment'' in America is without 
an ``e'' after the ``g''. Avoid the stilted expression ``one of us (R. B. 
G.) thanks $\ldots$''. Instead, try ``R. B. G. thanks$\ldots$''. Put sponsor 
acknowledgments in the unnumbered footnote on the first page.

\section*{References}

Please number citations consecutively within brackets \cite{b1}. The 
sentence punctuation follows the bracket \cite{b2}. Refer simply to the reference 
number, as in \cite{b3}---do not use ``Ref. \cite{b3}'' or ``reference \cite{b3}'' except at 
the beginning of a sentence: ``Reference \cite{b3} was the first $\ldots$''

Number footnotes separately in superscripts. Place the actual footnote at 
the bottom of the column in which it was cited. Do not put footnotes in the 
abstract or reference list. Use letters for table footnotes.

Unless there are six authors or more give all authors' names; do not use 
``et al.''. Papers that have not been published, even if they have been 
submitted for publication, should be cited as ``unpublished'' \cite{b4}. Papers 
that have been accepted for publication should be cited as ``in press'' \cite{b5}. 
Capitalize only the first word in a paper title, except for proper nouns and 
element symbols.

For papers published in translation journals, please give the English 
citation first, followed by the original foreign-language citation \cite{b6}.

\begin{thebibliography}{00}
  \bibitem{b1} United Nations, ``The Sustainable Development Goals Report 2022'' from \url{https://www.un.org/sustainabledevelopment/progress-report/}.
  \bibitem{b2} United Nations, ``Poverty - United Nations Sustainable Development''. Accessed February 5, 2023 from \url{https://www.un.org/sustainabledevelopment/poverty/}.
  \bibitem{b3} United Nations, ``Education - United Nations Sustainable Development''. Accessed February 5, 2023 from \url{https://www.un.org/sustainabledevelopment/education/}.

  
\bibitem{b4} I need a reference for the apprenticeship model.
\bibitem{b5} I need a reference for transfer learning.
\bibitem{b6} Y. Yorozu, M. Hirano, K. Oka, and Y. Tagawa, ``Electron spectroscopy studies on magneto-optical media and plastic substrate interface,'' IEEE Transl. J. Magn. Japan, vol. 2, pp. 740--741, August 1987 [Digests 9th Annual Conf. Magnetics Japan, p. 301, 1982].
\bibitem{b7} M. Young, The Technical Writer's Handbook. Mill Valley, CA: University Science, 1989.
\bibitem{b8} U.S. Department of Education - Institute of Education Sciences. ``Summary Tables, Institutional Characteristics and Student Charges, Price Trends - Massachusetts Institute of Technology'' Accessed May 8, 2023 from \url{https://nces.ed.gov/ipeds/SummaryTables/price-trend}
\end{thebibliography}
\vspace{12pt}
\color{red}
IEEE conference templates contain guidance text for composing and formatting conference papers. Please ensure that all template text is removed from your conference paper prior to submission to the conference. Failure to remove the template text from your paper may result in your paper not being published.

\end{document}